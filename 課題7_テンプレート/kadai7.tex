\documentclass[uplatex,dvipdfmx,a4j,12pt]{jsarticle}

\usepackage[utf8]{inputenc}
\usepackage{graphicx}
\usepackage{amsmath}
\usepackage{comment}
\usepackage{color}
\usepackage{url}
\usepackage{siunitx}
\usepackage[version=4]{mhchem}
\usepackage{paralist}
\usepackage{longtable}
\usepackage{multirow}
\usepackage[dvipdfmx]{hyperref}
\usepackage{pxjahyper}

\usepackage{enumitem}
\setlist[description]{parsep=5pt}
\setlist[enumerate]{parsep=5pt}

% 半角文字のパーセント、より右側は、コメントとして扱われます。

% タイトルページの内容をここで記述します。
\title{
  物理学実験IIレポート\\    % \\ は、強制的に改行するコマンドです。
  課題7 「γ線計測の基礎と応用」
  }
\author{
  実験回: 第x回 \\
  氏名: \\
  実験者番号:
  \\
  共同実験者:
  }
\date{
  実験日:2024年 x月 x日~ x月 x日 \\
  提出日:2024年 x月 x日}  % 実験日と、提出日を記入して下さい

% ここから本文が始まります。
\begin{document}

% 上で設定したタイトルページの情報は、\maketitle があって始めて、コンパイル後に表示されます。
\maketitle

% レポートのフィードバックのコメントを希望しない場合には、
% 48行目から55行目までをコメントアウト(行頭に%を入れる)。
% コメントを希望する場合は、何について聞きたいかを具体的に書いて下さい。

% 縦方向の空白を挿入するコマンドです。em は行の高さ分、を意味する単位です。
\vspace{2em}
\begin{center}
    \begin{minipage}{0.5\linewidth}
        レポートのコメントを希望します。

        具体的には、○○について評価を下さい。
    \end{minipage}
\end{center}

\vspace{5em}  


% 概要は論文・レポート全体を一つの段落にまとめたものです。
% 物理の分野では、概要では段落分けをしません。
%
\begin{abstract}
    \textcolor{red}{このテンプレートにある指示文章は全て削除し、自分で書いた文章に差し替えること。残っていた場合は読みやすさを損ねるため減点とする。}
    (概要ではレポートの概要を簡潔に記述せよ。例えば、以下のようなものである。)
    ○○の目的のために、■■の実験を行った。
    その結果△△であることが確かめられた。
\end{abstract}

% 強制的に改ページを行う
\newpage


\section{目的}
この実験のねらいが何であるかを、テキストを参考に簡潔にまとめる。


\section{原理}
実験装置の形状、寸法、性能などを簡潔に記載する。
ただし、工夫した部分がある場合などは明確に記述する。

\section{実験1}

\subsection{方法}

\subsection{結果}

\subsection{考察}


\section{実験2}

\subsection{方法}

\subsection{結果}

\subsection{考察}


\section{実験3}

\subsection{方法}

\subsection{結果}

\subsection{考察}



\section{設問への解答}


\section{結論}

結論の章では、実験目的を要約して記述し、それに対応する結論を最後に記述すること。

% 参考文献は、各課題に合わせて必要なものに書き換えること
\begin{thebibliography}{9}

    \bibitem{bunken1}
        A. Einstein, B. Podolsky, and N. Rosen, 
        Phys. Rev. \textbf{47}, (1935) 777.
        % いわゆるEPRパラドックスと呼ばれる量子力学に関する論文

    \bibitem{bunnken2}
        J.J. Aubert \textit{et al.},
        Phys. Lett. B \textbf{123} (1983) 275.
        % いわゆるEMC効果と呼ばれる現象についての実験の論文
    
\end{thebibliography}

\end{document}
