\documentclass[uplatex,dvipdfmx,a4j,12pt]{jsarticle}

\usepackage[utf8]{inputenc}
\usepackage{graphicx}
\usepackage{amsmath}
\usepackage{comment}
\usepackage{color}
\usepackage{url}
\usepackage{siunitx}
\usepackage[version=4]{mhchem}
\usepackage{paralist}
\usepackage{longtable}
\usepackage{multirow}
\usepackage[dvipdfmx]{hyperref}
\usepackage{pxjahyper}

\usepackage{enumitem}
\setlist[description]{parsep=5pt}
\setlist[enumerate]{parsep=5pt}

% 半角文字のパーセント、より右側は、コメントとして扱われます。

% タイトルページの内容をここで記述します。
\title{
  物理学実験IIレポート\\    % \\ は、強制的に改行するコマンドです。
  課題3 「原子分子のスペクトル」
  }
\author{
  実験回: 第x回 \\
  氏名: \\
  実験者番号:
  \\
  共同実験者:
  }
\date{
  実験日:2025年 x月 x日~ x月 x日 \\
  提出日:2025年 x月 x日}  % 実験日と、提出日を記入して下さい

% ここから本文が始まります。
\begin{document}

% 上で設定したタイトルページの情報は、\maketitle があって始めて、コンパイル後に表示されます。
\maketitle

% 縦方向の空白を挿入するコマンドです。em は行の高さ分、を意味する単位です。
\vspace{2em}

%レポートへのコメントを希望する場合は、その旨をここに記載してください。
\begin{center}
    \begin{minipage}{0.5\linewidth}
%        例; ○○について、コメントを希望します。
    \end{minipage}
\end{center}

\vspace{5em}  

% 概要は論文・レポート全体を簡潔にまとめたものです。
% 物理の分野では、概要では段落分けをしません。
\begin{abstract}
%    例; ○○の目的のために、■■の実験を行った。その結果△△であることが確かめられた。
\end{abstract}

% 強制的に改ページを行う
\newpage

%この課題全体の目的を簡潔に記述する。
\section{目的}

\newpage

% 以下の章立ては自由に変えてもよい。わかりやすくまとめること。

\section{分光検出器の特性}
%原理を自分の言葉で簡潔にまとめる。
\subsection{原理}

%方法について簡潔に記述する。テキストに書かれている手順の詳細を転記する必要は無い。
%たとえば、光学系の全体像、装置のメーカーと型番、光源の種類、試料の詳細、工夫したことなどを書くのが一般的。
\subsection{方法}

\subsection{光電子増倍管の特性}

%実験で得たデータの表やグラフを載せ、それらについて文章で説明する。
\subsubsection{結果}

%実験データに対して行った解析の内容や結果について説明する。
%調べたこと、考えたことを簡潔にまとめ、データに対する科学的な解釈を与える。
\subsubsection{考察}

\subsection{波長分解能}

\subsubsection{結果}

\subsubsection{考察}

\subsection{2次光の影響}

\subsubsection{結果}

\subsubsection{考察}

\subsection{波長較正と再現性}

\subsubsection{結果}

\subsubsection{考察}

%レポート問題1.1への解答
\subsection{レポート問題}

\newpage

\section{水素原子,アルカリ原子のスペクトル}
\subsection{原理}

\subsection{方法}

\subsection{水素原子のスペクトル}

\subsubsection{結果}

\subsubsection{考察}

\subsection{Na原子のスペクトル}

\subsubsection{結果}

\subsubsection{考察}

%レポート問題2.1、2.2への解答
\subsection{レポート問題}

\newpage

\section{分子のラマンスペクトル}
\subsection{原理}

\subsection{方法}

\subsection{励起レーザー波長の測定}

\subsubsection{結果}

\subsubsection{考察}

\subsection{二硫化炭素とベンゼンのラマンスペクトル}

\subsubsection{結果}

\subsubsection{考察}

%レポート問題3.1、3.2、3.3への解答
\subsection{レポート問題}

\newpage

%実験目的を要約して記述し、それに対応する結論を簡潔に記述する。
\section{結論}

\newpage

% 参考にした文献があれば記述する。以下の例を書き換えること。
\begin{thebibliography}{9}

    \bibitem{bunken1}
        A. Einstein, B. Podolsky, and N. Rosen, 
        Phys. Rev. \textbf{47}, (1935) 777.
        % いわゆるEPRパラドックスと呼ばれる量子力学に関する論文

    \bibitem{bunnken2}
        J.J. Aubert \textit{et al.},
        Phys. Lett. B \textbf{123} (1983) 275.
        % いわゆるEMC効果と呼ばれる現象についての実験の論文
    
\end{thebibliography}

\end{document}
