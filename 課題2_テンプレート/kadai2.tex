\documentclass[uplatex,dvipdfmx,a4j,12pt]{jsarticle}

\usepackage[utf8]{inputenc}
\usepackage{graphicx}
\usepackage{amsmath}
\usepackage{comment}
\usepackage{color}
\usepackage{url}
\usepackage{siunitx}
\usepackage[version=4]{mhchem}
\usepackage{paralist}
\usepackage{longtable}
\usepackage{multirow}
\usepackage[dvipdfmx]{hyperref}
\usepackage{pxjahyper}

\usepackage{enumitem}
\setlist[description]{parsep=5pt}
\setlist[enumerate]{parsep=5pt}

% 半角文字のパーセント、より右側は、コメントとして扱われます。

% タイトルページの内容をここで記述します。
\title{
  物理学実験 II レポート\\    % \\ は、強制的に改行するコマンドです。
  課題2 「光学」
  }
\author{
  実験回: 第x回 \\
  氏名: \\
  実験者番号: \\
\\
  共同実験者:
  }
  
\date{実験日:2025年 x月 x日~ x月 x日 \\
提出日:2025年 x月 x日}  % 実験日と、提出日を記入して下さい


% ここから本文が始まります。
\begin{document}

% 上で設定したタイトルページの情報は、\maketitle があって始めて、コンパイル後に表示されます。
\maketitle

% レポートのフィードバックのコメントを希望しない場合には、
% 51行目から58行目までをコメントアウト(行頭に%を入れる)。
% コメントを希望する場合は、何について聞きたいかを具体的に書いて下さい。

% 縦方向の空白を挿入するコマンドです。em は行の高さ分、を意味する単位です。
\vspace{2em}
\begin{center}
    \begin{minipage}{0.5\linewidth}
        % 締め切り後提出のレポートには基本的にはコメントを返信しません。
        % 詳細な採点結果等は返信することができません。
        例: レポートのコメントを希望します。

        具体的には、○○について評価を下さい。
    \end{minipage}
\end{center}

\vspace{5em}  


% 概要は論文・レポート全体を一つの段落にまとめたものです。
% 物理の分野では、概要では段落分けをしません。
\begin{abstract}
    \textcolor{red}{このテンプレートにある指示文章は全て削除し、自分で書いた文章に差し替えること。残っていた場合は読みやすさを損ねるため減点とする。}
    (概要ではレポートの概要を簡潔に記述せよ。例えば、以下のようなものである。)
    ○○の目的のために、■■の実験を行った。
    その結果△△であることが確かめられた。
\end{abstract}

% 強制的に改ページを行う
\newpage

% 提出するレポートでは、以下の指示文が掲載されないように、コメントアウトするか消すようにしてください。
% 指示文はじめ--------------------------------------------------
以下に記載されている書式はあくまで参考例の1つであり、必ずしもこの形式に従う必要はありません。レポート執筆者自身が実験結果や考察を整理しやすい形式、読者(教員)が読みやすい形式であればよいですが、\textbf{「目的」、「原理」、「方法」、「結果」、「考察」、「まとめ」}を(解いた場合には「課題」も)最低限含むようにしてください。

章立てについては、以下に示すような分け方がわかりやすいです。各実験や各テーマごとに原理、結果、方法、考察を記述すると、階層構造がわかりやすく、内容の整理がしやすいです。(目的についても、各項目ごとに記載して構いませんが、レポートの最初には必ず課題全体の背景や目的を簡潔に示してください。)

\subsubsection*{例1: 実験別で分ける場合}
    \begin{itemize}
        \item 実験1. ピンホールによる回折パターンの測定
        \item 実験2. フレネル回折
        \item 実験3. マイケルソン干渉計による分光
        \item 実験4. 計算ホログラフィ
        \item 実験5. 物体のホログラフィ
    \end{itemize}

\subsubsection*{例2: テーマ別で分ける場合}
    \begin{itemize}
        \item \S 1. 光波
        \item \S 2. 光の回折
        \item \S 3. 光の干渉とコヒーレンス
        \item \S 4. 空間光変調器を用いた回折実験
        \item \S 5. ホログラフィ
    \end{itemize}

\newpage
% 指示文おわり--------------------------------------------------

%簡潔に記述すること。
\section{目的}

% 以下の章立ては一例であり、自由に変えてもよいです。
\section{例: 実験1. ピンホールによる... or 光波}
\subsection{原理}

\subsection{方法}

\subsection{結果}

\subsection{考察}
\textcolor{red}{\textreferencemark この第2章のような章立てを各実験や各テーマごとに行ってください。}

\section{課題}

%簡潔に記述すること。
\section{結論}

\newpage

% レポート執筆に際して調べた文献を必ず記述すること。他にも参考にした文献があれば記述すること。
\begin{thebibliography}{9}

    \bibitem{bunken1}
        A. Einstein, B. Podolsky, and N. Rosen, 
        Phys. Rev. \textbf{47}, (1935) 777.
        % いわゆるEPRパラドックスと呼ばれる量子力学に関する論文

    \bibitem{bunnken2}
        J.J. Aubert \textit{et al.},
        Phys. Lett. B \textbf{123} (1983) 275.
        % いわゆるEMC効果と呼ばれる現象についての実験の論文
    
\end{thebibliography}

\end{document}
